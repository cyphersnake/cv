%============================================================================%
%
%	DOCUMENT DEFINITION
%
%============================================================================%

%we use article class because we want to fully customize the page and don't use a cv template
\documentclass[10pt,A4]{article}	

%----------------------------------------------------------------------------------------
%	ENCODING
%----------------------------------------------------------------------------------------

% we use utf8 since we want to build from any machine
\usepackage[utf8]{inputenc}		
\usepackage{tikz}

%----------------------------------------------------------------------------------------
%	LOGIC
%----------------------------------------------------------------------------------------

% provides \isempty test
\usepackage{xstring, xifthen}

%----------------------------------------------------------------------------------------
%	FONT BASICS
%----------------------------------------------------------------------------------------

% some tex-live fonts - choose your own

\usepackage[default]{raleway}

% set font default
\renewcommand*\familydefault{\sfdefault} 	
\usepackage[T1]{fontenc}

% more font size definitions
\usepackage{moresize}

%----------------------------------------------------------------------------------------
%	FONT AWESOME ICONS
%---------------------------------------------------------------------------------------- 

% include the fontawesome icon set
\usepackage{fontawesome}

% use to vertically center content
% credits to: http://tex.stackexchange.com/questions/7219/how-to-vertically-center-two-images-next-to-each-other
\newcommand{\vcenteredinclude}[1]{\begingroup
\setbox0=\hbox{\includegraphics{#1}}%
\parbox{\wd0}{\box0}\endgroup}

% use to vertically center content
% credits to: http://tex.stackexchange.com/questions/7219/how-to-vertically-center-two-images-next-to-each-other
\newcommand*{\vcenteredhbox}[1]{\begingroup
\setbox0=\hbox{#1}\parbox{\wd0}{\box0}\endgroup}

% icon shortcut
\newcommand{\icon}[3] { 							
	\makebox(#2, #2){\textcolor{maincol}{\csname fa#1\endcsname}}
}	

% icon with text shortcut
\newcommand{\icontext}[4]{ 						
	\vcenteredhbox{\icon{#1}{#2}{#3}}  \hspace{2pt}  \parbox{0.9\mpwidth}{\textcolor{#4}{#3}}
}

% icon with website url
\newcommand{\iconhref}[5]{ 						
    \vcenteredhbox{\icon{#1}{#2}{#5}}  \hspace{2pt} \href{#4}{\textcolor{#5}{#3}}
}

% icon with email link
\newcommand{\iconemail}[5]{ 						
    \vcenteredhbox{\icon{#1}{#2}{#5}}  \hspace{2pt} \href{mailto:#4}{\textcolor{#5}{#3}}
}

\usepackage{qrcode}

%----------------------------------------------------------------------------------------
%	PAGE LAYOUT  DEFINITIONS
%----------------------------------------------------------------------------------------

% page outer frames (debug-only)
% \usepackage{showframe}		

% we use paracol to display breakable two columns
\usepackage{paracol}

% define page styles using geometry
\usepackage[a4paper]{geometry}

% remove all possible margins
\geometry{top=1cm, bottom=1cm, left=1cm, right=1cm, paperheight=1383pt}

\usepackage{fancyhdr}
\pagestyle{empty}

% space between header and content
% \setlength{\headheight}{0pt}

% indentation is zero
\setlength{\parindent}{0mm}

%----------------------------------------------------------------------------------------
%	TABLE /ARRAY DEFINITIONS
%---------------------------------------------------------------------------------------- 

% extended aligning of tabular cells
\usepackage{array}

% custom column right-align with fixed width
% use like p{size} but via x{size}
\newcolumntype{x}[1]{%
>{\raggedleft\hspace{0pt}}p{#1}}%


%----------------------------------------------------------------------------------------
%	GRAPHICS DEFINITIONS
%---------------------------------------------------------------------------------------- 

%for header image
\usepackage{graphicx}

% use this for floating figures
% \usepackage{wrapfig}
% \usepackage{float}
% \floatstyle{boxed} 
% \restylefloat{figure}

%for drawing graphics		
\usepackage{tikz}				
\usetikzlibrary{shapes, backgrounds,mindmap, trees}

%----------------------------------------------------------------------------------------
%	Color DEFINITIONS
%---------------------------------------------------------------------------------------- 
\usepackage{transparent}
\usepackage{color}

% primary color
\definecolor{maincol}{RGB}{ 225, 0, 0 }

% accent color, secondary
% \definecolor{accentcol}{RGB}{ 250, 150, 10 }

% dark color
\definecolor{darkcol}{RGB}{ 70, 70, 70 }

% light color
\definecolor{lightcol}{RGB}{245,245,245}


% Package for links, must be the last package used
\usepackage[hidelinks]{hyperref}

% returns minipage width minus two times \fboxsep
% to keep padding included in width calculations
% can also be used for other boxes / environments
\newcommand{\mpwidth}{\linewidth-\fboxsep-\fboxsep}
	


%============================================================================%
%
%	CV COMMANDS
%
%============================================================================%

%----------------------------------------------------------------------------------------
%	 CV LIST
%----------------------------------------------------------------------------------------

% renders a standard latex list but abstracts away the environment definition (begin/end)
\usepackage{enumitem}
\setitemize[0]{topsep=1pt, partopsep=1pt, parsep=1pt, itemsep=1pt}
\newcommand{\cvlist}[1] {
	\begin{itemize}
    {#1}
    \end{itemize}
}

%----------------------------------------------------------------------------------------
%	 CV TEXT
%----------------------------------------------------------------------------------------

% base class to wrap any text based stuff here. Renders like a paragraph.
% Allows complex commands to be passed, too.
% param 1: *any
\newcommand{\cvtext}[1] {
	\begin{tabular*}
        {1\mpwidth}{p{0.98\mpwidth}}
		\parbox{1\mpwidth}{#1}
	\end{tabular*}
}

%----------------------------------------------------------------------------------------
%	CV SECTION
%----------------------------------------------------------------------------------------

% Renders a a CV section headline with a nice underline in main color.
% param 1: section title
\newcommand{\cvsection}[1] {
	\vspace{14pt}
	\cvtext{
		\textbf{\LARGE{\textcolor{darkcol}{\uppercase{#1}}}}\\[-4pt]
		\textcolor{maincol}{ \rule{0.1\textwidth}{2pt} } \\
	}
}

%----------------------------------------------------------------------------------------
%	META SKILL
%----------------------------------------------------------------------------------------

% Renders a progress-bar to indicate a certain skill in percent.
% param 1: name of the skill / tech / etc.
% param 2: level (for example in years)
% param 3: percent, values range from 0 to 1
\newcommand{\cvskill}[3] {
	\begin{tabular*}{1\mpwidth}{p{0.72\mpwidth}  r}
 		\textcolor{black}{\textbf{#1}} & \textcolor{maincol}{#2}\\
	\end{tabular*}%
	
	\hspace{4pt}
	\begin{tikzpicture}[scale=1,rounded corners=2pt,very thin]
		\fill [lightcol] (0,0) rectangle (1\mpwidth, 0.15);
		\fill [maincol] (0,0) rectangle (#3\mpwidth, 0.15);
  	\end{tikzpicture}%
}


%----------------------------------------------------------------------------------------
%	 CV EVENT
%----------------------------------------------------------------------------------------

% Renders a table and a paragraph (cvtext) wrapped in a parbox (to ensure minimum content
% is glued together when a pagebreak appears).
% Additional Information can be passed in text or list form (or other environments).
% the work you did
% param 1: time-frame i.e. Sep 14 - Jan 15 etc.
% param 2:	 event name (job position etc.)
% param 3: Customer, Employer, Industry
% param 4: Short description
% param 5: work done (optional)
% param 6: technologies include (optional)
% param 7: achievements (optional)
\newcommand{\cvevent}[7] {
	% we wrap this part in a parbox, so title and description are not separated on a pagebreak
	% if you need more control on page breaks, remove the parbox
	\parbox{\mpwidth} {
		\begin{tabular*}
            {1\mpwidth}
            {p{0.70\mpwidth}  r}
            \textcolor{black}
            {\Large{\textbf{#2}}} & \colorbox{maincol}{\makebox[0.25\mpwidth]{\textcolor{white}{#1}}} \\
			\textcolor{maincol}{\textbf{#3}} & \\
            \ifthenelse{\isempty{#4}} {} { 
                \parbox{1\mpwidth}{#4}
            }
        \end{tabular*}\\[8pt]
	}

	\ifthenelse{\isempty{#5}}{}{
		\vspace{2pt}
        \cvtext{{#5}}
	}

	\ifthenelse{\isempty{#6}}{}{
		\vspace{2pt}
        \textbf{\space \space Projects:}
		{#6}
	}

	\ifthenelse{\isempty{#7}}{}{
		\vspace{2pt}
        \textbf{\space \space Achievements:}
		{#7}
	}
	\vspace{14pt}
}

%----------------------------------------------------------------------------------------
%	 CV META EVENT
%----------------------------------------------------------------------------------------

% Renders a CV event on the sidebar
% param 1: title
% param 2: subtitle (optional)
% param 3: customer, employer, etc,. (optional)
% param 4: info text (optional)
\newcommand{\cvmetaevent}[4] {
	\textcolor{maincol} {\cvtext{\textbf{\begin{flushleft}#1\end{flushleft}}}}

	\ifthenelse{\isempty{#2}}{}{
	\textcolor{darkcol} {\cvtext{\textbf{#2}} }
	}

	\ifthenelse{\isempty{#3}}{}{
		\cvtext{{ \textcolor{darkcol} {#3} }}\\
	}

	\cvtext{#4}\\[14pt]
}

%============================================================================%
%
%
%
%	DOCUMENT CONTENT
%
%
%
%============================================================================%
\begin{document}
\columnratio{0.31}
\setlength{\columnsep}{2.2em}
\setlength{\columnseprule}{4pt}
\colseprulecolor{lightcol}
\begin{paracol}{2}
\begin{leftcolumn}
%---------------------------------------------------------------------------------------
%	META IMAGE
%----------------------------------------------------------------------------------------
\includegraphics[width=\linewidth]{michail_gorbachev.jpg}	%trimming relative to image size

%---------------------------------------------------------------------------------------
%	META SKILLS
%----------------------------------------------------------------------------------------

\cvsection{\faGlobe\space CONTACT}
	
\icontext{MapMarker}{12}{Tagansky District\\ Moscow, Russia}{black}\\[6pt]
\iconhref{MobilePhone}{12}{+7 999 970 2019}{tel:+79999702019}{black}\\[6pt]
\iconemail{Envelope}{12}{michail.gorbachev@pm.me}{michail.gorbachev@pm.me}{black}\\[6pt]
\iconhref{Send}{12}{@sadsnake}{https://t.me/sadsnake}{black}\\[6pt]
\iconhref{Github}{12}{sadsnake42}{https://github.com/sadsnake42}{black}\\[6pt]

\cvsection{\faFlash\space SKILLS}

\cvskill{Rust} {Strong} {1} \\[+1pt]
\cvskill{C\textbackslash C++} {Strong} {1} \\[+1pt]
\cvskill{Arch Linux} {Strong} {1} \\[+1pt]
\cvskill{Software Architecture} {Fine} {0.85} \\[+1pt]
\cvskill{Assembler} {Fine} {0.75} \\[+1pt]
\cvskill{Python} {Fine} {0.75} \\[+1pt]
\cvskill{Go} {Low} {0.55} \\[+1pt]

%---------------------------------------------------------------------------------------
%	EDUCATION
%----------------------------------------------------------------------------------------
\cvsection{\faGraduationCap\space EDUCATION}

\cvmetaevent
{2016 - 2020}
{Information Security \\ Master's Degree }
{\href{https://bmstu.ru/en/}{Bauman Moscow State Technical University}}
{
    My education program includes deep math knowledge and advanced computer science topics. \\
    My graduating project:\\ \textit{''Development of an algorithm for searching for sensitive commands in dynamic analysis of the security of program code for RISC family architecture processors''}
}

\cvsection{\faAngellist\space Hobby}
\cvlist {
    \item \faStreetView\space Hiking \& Cycling
    \item \faBook\space Political Science
    \item \faUsers\space \href{https://en.wikipedia.org/wiki/Mafia_(party_game)}{Social Game Mafia}
}

\vfill\null
%\cvqrcode{0.7}

\end{leftcolumn}
\begin{rightcolumn}
%---------------------------------------------------------------------------------------
%	TITLE  HEADER
%----------------------------------------------------------------------------------------
\fcolorbox{white}{darkcol}{\begin{minipage}[c][3cm][c]{1\mpwidth}
    \begin {center}
        \HUGE{ \textbf{ \textcolor{white}{ \uppercase{ MICHAIL GORBACHEV } } } } \\[-24pt]
        \textcolor{white}{ \rule{0.1\textwidth}{1.25pt} } \\[4pt]
        \large{ \textcolor{white} { Rust Software Engineer } }
        \\
        { \includegraphics[scale=0.025]{cuddlyferris.png} }
	\end {center}
\end{minipage}} \\[14pt]
\vspace{-12pt}

%---------------------------------------------------------------------------------------
%	PROFILE
%----------------------------------------------------------------------------------------
\cvsection{PROFILE}

\cvtext {
    I'm Rust Developer from Moscow. \\
    I love my job and I want to improve my skills among professionals with a team that values quality, productivity and code cleanliness. For my part, I can promise high standards for software implementation, flexibility in solving business issues, and a wide range of competencies, from blockchain cryptography to processor architecture.

    \iffalse
    \bigskip
    #TODO
    An important question for me is having a code review, as this allows you to expand the competencies of all team members and makes the code base of high quality.
    \fi
}

%---------------------------------------------------------------------------------------
%	WORK EXPERIENCE
%----------------------------------------------------------------------------------------
%\vfill\null
\cvsection{WORK EXPERIENCE}

\cvevent
    {Sep 20 - NOW}
    {\href{https://www.joinsprouttherapy.com/}{Sprout Therapy}}
    {Software Architect}
    {\small{The tech-forward provider of in-home and online Applied Behavior Analysis (ABA) Therapy.}}
    {  }
    { 
        \cvlist{ \item I work on the creation of architecture, maintain documentation, decompose business requests. 
    }}
    { \cvlist{
        \item Helped the company to put things in order after the MVP state.
        \item Decomposed the monolith into services architecture. 
        \item Improved development infrastructure 
    }}

\cvevent
    {May 20 - Sep 20}
    {\href{https://www.group-ib.com/}{Group-IB}}
    {Rust\textbackslash Python Software Engineer}
    {One of the leading providers of solutions aimed at detection and prevention of cyberattacks, online fraud, and IP protection. }
    {
    }
    {
        \begin{itemize} 
            \item Information security software in Rust
            \item Front-end in Rust with \underline{\href{https://github.com/yewstack/yew}{Yew}}
            \item Tools for analytics in Python
            \item Work with internal information security system
        \end{itemize}
    }
    {
        \cvlist{ \item Improve work-flow of companies analytics \item Improved the quality of the codebase of analyst utilities \item Introduced linters practice}
    }

\cvevent
    {Nov 19 - May 20}
    {\href{https://mixbytes.io/}{MixBytes}}
    {Rust Blockchain Developer}
    {\small{The team of experienced developers providing top-notch blockchain solutions, smart contract security audits and tech advisory.}}
    {
       The first time, working as a solo substrate developer with DeFi projects, realizing few POC projects, and work with analytics for stable coin projects. 
       Subsequently, combined with team development within the framework of EOS and training new Rust developers.
    }
    {
        \cvlist {
            \item \underline{\href{https://github.com/mixbytes/substrate-oracle-parachain}{Oracle}}
            \item Full analytic and MVP for Stable Coin project
            \item Improvement EOS gambling contracts
        }
    }
    {
        \cvlist{
            \item Created development analytics for the stable coin project
            \item Realize few DeFi POC
        }
    }

\cvevent
    {May 19 - Nov 19}
    {\href{https://ntprogress.com/}{NT Progress}}
    {C++ Software Developer}
    {\small{The modern high-tech company with the large-scale developments in different business areas, advanced infrastructure and a wide range of services.}}
    { 
        Worked in a business team, dealing with the full cycle of developing user stories.
        However, over time, due to high technical experience, switched to solving service problems: 
            refactoring problem areas,
            improving infrastructure,
            logging,
            tests, etc
    }
    { % Projects
        \cvlist {
            \item Work with various business inquiries
            \item Involve in reducing the company's technical debt
            \item Python auto-tests infrastructure
        }
    }
    { % Achievements
        \cvlist {
            \item Drastically accelerated (~5x faster) the approve process 
        }
    }

\cvevent
    {Jun 15 - May 19}
    {\href{https://www.ispras.ru/en/}{Institute for System Programming of the Russian Academy of Science}}
    {C++\textbackslash Rust Developer }
    {The research organization specializing in systems programming. \\ \small{Department of low level development and analytics} }
    {
        The first time I worked on supporting and improving the legacy project (8+ years) C++\textbackslash QT,
        after the decision to rewrite the project - as a Rust developer.
    }
    {
        \cvlist {
            \item \href{https://www.ispras.ru/en/publications/2011/features_of_tral_a_binary_code_analysis_framework_and_its_future_directions/}{Trall}
            \item Disassemble tools with unified algorithm for many architecture (x86, RISC, Arm etc).
            \item Develop and implement data analysis algorithms based on a dynamic execution track
        }
    }
    {
        \cvlist {
            \item Implement code coverage and dynamic code checker for C++ legacy source code based on gcov
            \item Implement many algorithms related to the analysis of disassembled code
            \item Rewrite big code base of application from C++ to Rust
        }
    }

\end{rightcolumn}
\end{paracol}

\end{document}
